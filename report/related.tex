\section{Related Work}
For linear systems with white noise, Kalman filter (KF)~\cite{kalman,prob} is the optimal Bayesian filter algorithm and is used popularly. But KF doesn't work well for non-linear systems. For non-linear systems, the extended Kalman filter (EKF)~\cite{prob} is one of the most popular filtering algorithms. EKF is usually implemented using a first order Taylor expansion, but higher order EKFs have also been used. Although the EKF (in its many forms) is a widely used filtering strategy, over several decades of experience with it has led to a general consensus within the tracking and control community that it is difficult to implement, difficult to tune, and only reliable for systems which are almost linear on the time scale of the update intervals~\cite{ukf1}. We implemented KF and EKF for assignment 4 ball tracking.

Unscented Kalman filter (UKF) is an extension of KF and EKF. It was introduced by~\cite{ukf1,ukf3} with the accuracy of EKF with second-order Taylor approximation, and improved by~\cite{ukf2} to make its accuracy comparable to third order. In UKF, a set of appropriately chosen weighted points are used to parameterize the means and covariances of the probability distributions. UKF has the same time complexity as first-order EKF while mitigating the sub-optimal performance and possible divergence.

Another aimilar algorithm is the Gaussian sum filter introduced in~\cite{gpf}. In this method, there is a weighted sum of multiple non-correlated Gaussian particles. As a sum of Gaussian distributions is a Gaussian distribution, the estimated distribution is also Gaussian. In the paper, they show that under the Gaussianity assumption, the Gaussian particle filter is asymptotically optimal in the number of particles and, hence, has much-improved performance and versatility over other Gaussian filters, especially when nontrivial nonlinearities are present. On the other hand, it has lower complexity than particle filters.

The simplest non-parametric technique is the grid (histogram) algorithm. An extension to grid algorithm is the dynamic grid algorithm~\cite{grid1,prob}. In this algorithm, the grid is dynamically resized if the probability mass in the grid goes above a certain threshold. We implement these algorithms along with particle filter, they will be described in detail in the next section.

A different approach for filtering that we looked into is where the belief distribution is stored using a neural network. \cite{neuspike1} mention that organisms acting in uncertain dynamical environments often employ exact or approximate Bayesian statistical calculations in order to continuously estimate the environmental state, integrate information from multiple sensory modalities, form predictions and choose actions. They study how these computations could have been implemented in the neural network, and introduce the spike neural networks. Other relevant works used a recurrent neural network (RNN) to learn the internal behavior of the dynamic system, and a feedforward neural network attached to the RNN that outputs the estimated state~\cite{neu5,neu1,neu2}. Another paper~\cite{neu4} used a multilayer network that approximates the posterior, and gives output the state estimate when subjected to unknown noises.

Several analytical techniques are also used for filtering~\cite{anl1,anl2,anl3}. We mentioned only a few papers, but there is an enormous literature related to filtering, estimation theory, and their use in probabilistic robotics. 